\documentclass{article}

\usepackage[utf8]{inputenc}
\usepackage{fancyhdr}
\usepackage{extramarks}
\usepackage{amsmath}
\usepackage{amsthm}
\usepackage{amsfonts}
\usepackage{tikz}
\usepackage[plain]{algorithm}
\usepackage{algpseudocode}
\usepackage{listings}

\usetikzlibrary{automata,positioning}

%
% Basic Document Settings
%

\topmargin=-0.45in
\evensidemargin=0in
\oddsidemargin=0in
\textwidth=6.5in
\textheight=9.0in
\headsep=0.25in

\linespread{1.1}

\pagestyle{fancy}
\lhead{\hmwkAuthorName}
%\chead{\hmwkClass\ (\hmwkClassInstructor\ \hmwkClassTime): \hmwkTitle}
\rhead{\firstxmark}
\lfoot{\lastxmark}
\cfoot{\thepage}

\renewcommand\headrulewidth{0.4pt}
\renewcommand\footrulewidth{0.4pt}

\setlength\parindent{0pt}

%
% Create Problem Sections
%

\newcommand{\enterProblemHeader}[1]{
    \nobreak\extramarks{}{Problem \arabic{#1} continued on next page\ldots}\nobreak{}
    \nobreak\extramarks{Problem \arabic{#1} (continued)}{Problem \arabic{#1} continued on next page\ldots}\nobreak{}
}

\newcommand{\exitProblemHeader}[1]{
    \nobreak\extramarks{Problem \arabic{#1} (continued)}{Problem \arabic{#1} continued on next page\ldots}\nobreak{}
    \stepcounter{#1}
    \nobreak\extramarks{Problem \arabic{#1}}{}\nobreak{}
}

\setcounter{secnumdepth}{0}
\newcounter{partCounter}
\newcounter{homeworkProblemCounter}
\setcounter{homeworkProblemCounter}{1}
\nobreak\extramarks{Problem \arabic{homeworkProblemCounter}}{}\nobreak{}

%
% Homework Problem Environment
%
% This environment takes an optional argument. When given, it will adjust the
% problem counter. This is useful for when the problems given for your
% assignment aren't sequential. See the last 3 problems of this template for an
% example.
%
\newenvironment{homeworkProblem}[1][-1]{
    \ifnum#1>0
        \setcounter{homeworkProblemCounter}{#1}
    \fi
    \section{Problem \arabic{homeworkProblemCounter}}
    \setcounter{partCounter}{1}
    \enterProblemHeader{homeworkProblemCounter}
}{
    \exitProblemHeader{homeworkProblemCounter}
}

%
% Homework Details
%   - Title
%   - Due date
%   - Class
%   - Section/Time
%   - Instructor
%   - Author
%

\newcommand{\hmwkTitle}{Homework\ \#7}
\newcommand{\hmwkDueDate}{April 05, 2019}
\newcommand{\hmwkClass}{CPE 301}
\newcommand{\hmwkClassTime}{Section 101}
\newcommand{\hmwkClassInstructor}{Dr. Dwight Egbert}
\newcommand{\hmwkAuthorName}{\textbf{Michael DesRoches}}

%
% Title Page
%

\title{
    \vspace{2in}
    \textmd{\textbf{\hmwkClass:\ \hmwkTitle}}\\
    \normalsize\vspace{0.1in}\small{Due\ on\ \hmwkDueDate\ at 9:00am}\\
    \vspace{0.1in}\large{\textit{\hmwkClassInstructor\ \hmwkClassTime}}
    \vspace{3in}
}

\author{\hmwkAuthorName}
\date{}

\renewcommand{\part}[1]{\textbf{\large Part \Alph{partCounter}}\stepcounter{partCounter}\\}


\begin{document}
\maketitle
\pagebreak

%\begin{homeworkProblem}

\begin{lstlisting}

Description of Purpose:
This homework gets us familiar with analog-to-digital converters. It's a way to
bridge the gap in the analog environment. Potentiometers, Thermistors,
accelerometers, and ambient light sensors are a few examples of those devices.
We will set the analog inputs as neccessary using the arduino registers to
further comprehend this ordeal.


int16_t adc_read(uint8_t mux)
{
    uint8_t low;

    ADCSRA = (1<<ADEN) | ADC_PRESCALER;             // enable ADC
    ADCSRB = (1<<ADHSM) | (mux & 0x20);             // high speed mode
    ADMUX = aref | (mux & 0x1F);                    // configure mux input
    ADCSRA = (1<<ADEN) | ADC_PRESCALER | (1<<ADSC); // start the conversion
    while (ADCSRA & (1<<ADSC)) ;                    // wait for result
    low = ADCL;                                     // must read LSB first
    return (ADCH << 8) | low;                       // must read MSB only once!
}
void adc_start(uint8_t mux, uint8_t aref)
{
    ADCSRA = (1<<ADEN) | ADC_PRESCALER;     // enable the ADC, interrupt disabled
    ADCSRB = (1<<ADHSM) | (mux & 0x20);
    ADMUX = aref | (mux & 0x1F);            // configure mux and ref
    head = 0;                               // clear the buffer
    tail = 0;                               // and then begin auto trigger mode
    ADCSRA = (1<<ADSC) | (1<<ADEN) | (1<<ADATE) | (1<<ADIE) | ADC_PRESCALER;
    sei();
}
int16_t adc_read(void)
{
    uint8_t h, t;
    int16_t val;

    do {
        h = head;
        t = tail;                   // wait for data in buffer
    } while (h == t);
    if (++t >= BUFSIZE) t = 0;
    val = buffer[t];                // remove 1 sample from buffer
    tail = t;
    return val;
}


\end{lstlisting}

  \textbf{Solution}
%\end(homeworkProblem)
\end{document}

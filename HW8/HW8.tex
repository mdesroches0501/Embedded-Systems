\documentclass{article}

\usepackage[utf8]{inputenc}
\usepackage{fancyhdr}
\usepackage{extramarks}
\usepackage{amsmath}
\usepackage{amsthm}
\usepackage{amsfonts}
\usepackage{tikz}
\usepackage[plain]{algorithm}
\usepackage{algpseudocode}
\usepackage{listings}

\usetikzlibrary{automata,positioning}

%
% Basic Document Settings
%

\topmargin=-0.45in
\evensidemargin=0in
\oddsidemargin=0in
\textwidth=6.5in
\textheight=9.0in
\headsep=0.25in

\linespread{1.1}

\pagestyle{fancy}
\lhead{\hmwkAuthorName}
%\chead{\hmwkClass\ (\hmwkClassInstructor\ \hmwkClassTime): \hmwkTitle}
\rhead{\firstxmark}
\lfoot{\lastxmark}
\cfoot{\thepage}

\renewcommand\headrulewidth{0.4pt}
\renewcommand\footrulewidth{0.4pt}

\setlength\parindent{0pt}

%
% Create Problem Sections
%

\newcommand{\enterProblemHeader}[1]{
    \nobreak\extramarks{}{Problem \arabic{#1} continued on next page\ldots}\nobreak{}
    \nobreak\extramarks{Problem \arabic{#1} (continued)}{Problem \arabic{#1} continued on next page\ldots}\nobreak{}
}

\newcommand{\exitProblemHeader}[1]{
    \nobreak\extramarks{Problem \arabic{#1} (continued)}{Problem \arabic{#1} continued on next page\ldots}\nobreak{}
    \stepcounter{#1}
    \nobreak\extramarks{Problem \arabic{#1}}{}\nobreak{}
}

\setcounter{secnumdepth}{0}
\newcounter{partCounter}
\newcounter{homeworkProblemCounter}
\setcounter{homeworkProblemCounter}{1}
\nobreak\extramarks{Problem \arabic{homeworkProblemCounter}}{}\nobreak{}

%
% Homework Problem Environment
%
% This environment takes an optional argument. When given, it will adjust the
% problem counter. This is useful for when the problems given for your
% assignment aren't sequential. See the last 3 problems of this template for an
% example.
%
\newenvironment{homeworkProblem}[1][-1]{
    \ifnum#1>0
        \setcounter{homeworkProblemCounter}{#1}
    \fi
    \section{Problem \arabic{homeworkProblemCounter}}
    \setcounter{partCounter}{1}
    \enterProblemHeader{homeworkProblemCounter}
}{
    \exitProblemHeader{homeworkProblemCounter}
}

%
% Homework Details
%   - Title
%   - Due date
%   - Class
%   - Section/Time
%   - Instructor
%   - Author
%

\newcommand{\hmwkTitle}{Homework\ \#8}
\newcommand{\hmwkDueDate}{April 8, 2019}
\newcommand{\hmwkClass}{CPE 301}
\newcommand{\hmwkClassTime}{Section 101}
\newcommand{\hmwkClassInstructor}{Dr. Dwight Egbert}
\newcommand{\hmwkAuthorName}{\textbf{Michael DesRoches}}

%
% Title Page
%

\title{
    \vspace{2in}
    \textmd{\textbf{\hmwkClass:\ \hmwkTitle}}\\
    \normalsize\vspace{0.1in}\small{Due\ on\ \hmwkDueDate\ at 9:00am}\\
    \vspace{0.1in}\large{\textit{\hmwkClassInstructor\ \hmwkClassTime}}
    \vspace{3in}
}

\author{\hmwkAuthorName}
\date{}

\renewcommand{\part}[1]{\textbf{\large Part \Alph{partCounter}}\stepcounter{partCounter}\\}


\begin{document}
\maketitle
\pagebreak

\begin{homeworkProblem}

\begin{lstlisting}
Description of Purpose:
Homework 8 helps us get more familiar with the arduino's language by having us
do more program's. As far of the scope of what the homework wants us to do, I'm
not sure until I actually do the assignments. But, from the details of what I
can read, it seems that we are getting more familiar with masking, DDR, Port,
and PIN operations.


Read chapter 8 of the textbook and browse section 26 of the Atmel 2560’s datasheet, paying extra
attention to sections 26.2, 26.3, 26.4, and 26.8.
\end{lstlisting}
  \textbf{Solution}


\end{homeworkProblem}
\pagebreak

\begin{homeworkProblem}

\begin{lstlisting}
The international tuning standard for musical instruments is A above middle C
at a frequency of 440Hz. Write an Arduino Mega C language program to generate
this tuning frequency and sound a 440 Hz tone on a loudspeaker connected to
PortB.6 using Timer 1.
\end{lstlisting}

  \textbf{Solution}


\end{homeworkProblem}
\pagebreak


\begin{homeworkProblem}
\begin{lstlisting}
Write an Arduino Mega C language program using the Arduino ATmega2560 timer1 in
Normal mode to generate a 12 kHz square wave on PortB.6 using Timer 1.
\end{lstlisting}

  \textbf{Solution}

\end{homeworkProblem}

\pagebreak

\begin{homeworkProblem}

\begin{lstlisting}
Write an Arduino Mega C language program to generate a 500Hz signal on PortB.6
using Timer 1 in Normal mode. The wave should have a 30% duty cycle (duty cycle
= high time / period).
\end{lstlisting}

  \textbf{Solution}

\end{homeworkProblem}
\pagebreak


\end{document}

\documentclass{article}

\usepackage[utf8]{inputenc}
\usepackage{fancyhdr}
\usepackage{extramarks}
\usepackage{amsmath}
\usepackage{amsthm}
\usepackage{amsfonts}
\usepackage{tikz}
\usepackage[plain]{algorithm}
\usepackage{algpseudocode}
\usepackage{listings}

\usetikzlibrary{automata,positioning}

%
% Basic Document Settings
%

\topmargin=-0.45in
\evensidemargin=0in
\oddsidemargin=0in
\textwidth=6.5in
\textheight=9.0in
\headsep=0.25in

\linespread{1.1}

\pagestyle{fancy}
\lhead{\hmwkAuthorName}
%\chead{\hmwkClass\ (\hmwkClassInstructor\ \hmwkClassTime): \hmwkTitle}
\rhead{\firstxmark}
\lfoot{\lastxmark}
\cfoot{\thepage}

\renewcommand\headrulewidth{0.4pt}
\renewcommand\footrulewidth{0.4pt}

\setlength\parindent{0pt}

%
% Create Problem Sections
%

\newcommand{\enterProblemHeader}[1]{
    \nobreak\extramarks{}{Problem \arabic{#1} continued on next page\ldots}\nobreak{}
    \nobreak\extramarks{Problem \arabic{#1} (continued)}{Problem \arabic{#1} continued on next page\ldots}\nobreak{}
}

\newcommand{\exitProblemHeader}[1]{
    \nobreak\extramarks{Problem \arabic{#1} (continued)}{Problem \arabic{#1} continued on next page\ldots}\nobreak{}
    \stepcounter{#1}
    \nobreak\extramarks{Problem \arabic{#1}}{}\nobreak{}
}

\setcounter{secnumdepth}{0}
\newcounter{partCounter}
\newcounter{homeworkProblemCounter}
\setcounter{homeworkProblemCounter}{1}
\nobreak\extramarks{Problem \arabic{homeworkProblemCounter}}{}\nobreak{}

%
% Homework Problem Environment
%
% This environment takes an optional argument. When given, it will adjust the
% problem counter. This is useful for when the problems given for your
% assignment aren't sequential. See the last 3 problems of this template for an
% example.
%
\newenvironment{homeworkProblem}[1][-1]{
    \ifnum#1>0
        \setcounter{homeworkProblemCounter}{#1}
    \fi
    \section{Problem \arabic{homeworkProblemCounter}}
    \setcounter{partCounter}{1}
    \enterProblemHeader{homeworkProblemCounter}
}{
    \exitProblemHeader{homeworkProblemCounter}
}

%
% Homework Details
%   - Title
%   - Due date
%   - Class
%   - Section/Time
%   - Instructor
%   - Author
%

\newcommand{\hmwkTitle}{Homework\ \#10}
\newcommand{\hmwkDueDate}{April 19, 2019}
\newcommand{\hmwkClass}{CPE 301}
\newcommand{\hmwkClassTime}{Section 101}
\newcommand{\hmwkClassInstructor}{Dr. Dwight Egbert}
\newcommand{\hmwkAuthorName}{\textbf{Michael DesRoches}}

%
% Title Page
%

\title{
    \vspace{2in}
    \textmd{\textbf{\hmwkClass:\ \hmwkTitle}}\\
    \normalsize\vspace{0.1in}\small{Due\ on\ \hmwkDueDate\ at 9:00am}\\
    \vspace{0.1in}\large{\textit{\hmwkClassInstructor\ \hmwkClassTime}}
    \vspace{3in}
}

\author{\hmwkAuthorName}
\date{}

\renewcommand{\part}[1]{\textbf{\large Part \Alph{partCounter}}\stepcounter{partCounter}\\}


\begin{document}
\maketitle
\pagebreak

\begin{homeworkProblem}

\begin{lstlisting}
Description of Purpose:
Homework 10 has us working with VERILOG and VHDL. After loooking at some
programming tutorials, we are to evaluate which one is more usefull. Also, we
are to look at the brief history of both to further understand the concepts of
both.

What is the purpose of an interrupt?
\end{lstlisting}

  \textbf{Solution}
  \\Interrupt is defined as it's a signal which is generated from devices joined to \\
  a pc or from a program inside the PC that requires the working framework to \\
  interrupt it and make sense of what to do next.\\

\end{homeworkProblem}
%\pagebreak

\begin{homeworkProblem}

\begin{lstlisting}
Describe the flow of events when an interrupt occurs.
\end{lstlisting}

  \textbf{Solution}


\end{homeworkProblem}
\pagebreak


\begin{homeworkProblem}
\begin{lstlisting}
Describe the interrupt features available with the Atmega328P.
\end{lstlisting}

  \textbf{Solution}

\end{homeworkProblem}

\pagebreak

\begin{homeworkProblem}

\begin{lstlisting}
What is interrupt priority? How is it determined?
\end{lstlisting}

  \textbf{Solution}

\end{homeworkProblem}
\pagebreak

begin{homeworkProblem}

\begin{lstlisting}
What steps are required by the system designer to properly configure an
interrupt?
\end{lstlisting}

  \textbf{Solution}

\end{homeworkProblem}
\pagebreak

begin{homeworkProblem}

\begin{lstlisting}
How is the interrupt system turned "on" and "off"?
\end{lstlisting}

  \textbf{Solution}

\end{homeworkProblem}
\pagebreak

begin{homeworkProblem}

\begin{lstlisting}
Write a program to set up timer1 using NORMAL mode so that it generates an
interrupt in exactly 1/8 of a second. Write an interrupt service routine (ISR),
triggered by the timer interrupt TOV1 that stops, resets, and restarts the
timer and toggles the Arduino Mega LED each time it is called. This will
produce a light that blinks 4 times/sec.
\end{lstlisting}

  \textbf{Solution}

\end{homeworkProblem}
\pagebreak

\begin{homeworkProblem}

\begin{lstlisting}
Write a program to set up timer1 using NORMAL mode so that it generates an
interrupt in exactly 1/8 of a second. Write an interrupt service routine
(ISR), triggered by the timer interrupt TOV1 that stops, resets, and restarts
the timer and toggles the Arduino Mega LED each time it is called. This will
produce a light that blinks 4 times/sec.
\end{lstlisting}

  \textbf{Solution}
Interrupt is defined as it's a signal which is generated from devices joined to \\
a pc or from a program inside the PC that requires the working framework to \\
interrupt it and make sense of what to do next.
\end{homeworkProblem}
\pagebreak
\end{lstlisting}

  \textbf{Solution}

\end{homeworkProblem}
\pagebreak

\end{document}
